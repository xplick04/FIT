\documentclass[a4paper,twocolumn,11pt]{article}
\usepackage[utf8]{inputenc}
\usepackage[left=1.5cm,text={18cm,25cm},top=2.5cm]{geometry}
\usepackage[czech]{babel}
\usepackage[IL2]{fontenc}
\usepackage{amsmath}
\usepackage{amsthm}
\usepackage{amssymb}
\usepackage{amstext}
\usepackage[unicode, hidelinks]{hyperref}
\usepackage{times}

\newtheorem{definice1}{Definice}
\newtheorem{veta}{Věta}

\begin{document}
\begin{titlepage}
\begin{center}
\Huge
\textsc{FAKULTA INFORMAČNÍCH TECHNOLOGIÍ\\
VYSOKÉ UČENÍ TECHNICKÉ V BRNĚ
}
\vspace{\stretch{0,381966}}
\\
\LARGE
Typografie a publikování – 2. projekt\\
Sazba dokumentů a matematických výrazů
\vspace{\stretch{0,618034}}
\end{center}
{\LARGE 2021 \hfill
Maxim Plička(xplick04)}

\end{titlepage}
\section*{Úvod}
V této úloze si vyzkoušíme sazbu titulní strany, matematických vzorců, 
prostředí a dalších textových struktur 
obvyklých pro technicky zaměřené texty (například rovnice (\ref{rovnice1})
nebo Definice \ref{def1} na straně \pageref{def1}). Rovněž si vyzkoušíme používání odkazů \verb!\ref! a \verb!\pageref!.

Na titulní straně je využito sázení nadpisu podle optického středu s využitím zlatého řezu.
Tento postup byl probírán na přednášce. Dále je použito odřádkování se
zadanou relativní velikostí 0.4 em a 0.3 em.

V případě, že budete potřebovat vyjádřit matematickou
konstrukci nebo symbol a nebude se Vám dařit jej nalézt
v samotném \LaTeX u, doporučuji prostudovat možnosti balíku maker \AmS-\LaTeX.

\section{Matematický text}
Nejprve se podíváme na sázení matematických symbolů
a výrazů v plynulém textu včetně sazby definic a vět s využitím balíku \texttt{amsthm}. Rovněž použijeme poznámku pod
čarou s použitím příkazu \verb!\footnote!. Někdy je vhodné
použít konstrukci \verb!\mbox{}!, která říká, že text nemá být
zalomen.

\begin{definice1}
\label{def1}
\textnormal{Rozšířený zásobníkový automat} (RZA) je definován jako sedmice tvaru $A = (Q, \Sigma, \Gamma, \delta, q_0, Z_0, F)$,
kde:
\begin{itemize}
\item $Q$ je konečná množina \textnormal{vnitřních (řídicích) stavů,}
\item $\Sigma$ je konečná \textnormal{vstupní abeceda,}
\item $\Gamma$ je konečná \textnormal{zásobníková abeceda,}
\item $\delta$ je \textnormal{přechodová funkce} 
$Q\times(\Sigma\cup\{\epsilon\})\times\Gamma^*\rightarrow2^{Q\times\Gamma^*}$,

\item $q_0 \in Q$ je \textnormal{počáteční stav},$Z_0 \in \Gamma$ \textnormal{je startovací symbol zásobníku} a $F \subseteq Q$ je množina \textnormal{koncových stavů.}
\end{itemize}
\end{definice1}
Nechť $P = (Q, \Sigma, \Gamma, \delta, q_0, Z_0, F)$ je rozšířený zásobníkový automat. \textit{Konfigurací} nazveme trojici $(q,w,\alpha)\in Q\times\Sigma^*\times\Gamma^*$, kde $q$ je aktuální stav vnitřního řízení,
$w$ je dosud nezpracovaná část vstupního řetězce a $\alpha = Z_{i_1} Z_{i_2} \dots Z_{i_k}$
je obsah zásobníku\footnote{$Z_{i1}$ je vrchol zásobníku}.

\subsection{Podsekce obsahující větu a odkaz}
\begin{definice1}
\label{def2}
\textnormal{Řetězec $w$ nad abecedou $\Sigma$ je přijat RZA}
A jestliže $(q_0, w, Z_0) \underset{A}{\stackrel{*}\vdash} (q_F, \epsilon, \gamma)$ pro nějaké $\gamma\in\Gamma^*$ a $q_F \in F$.
Množinu $L(A) = \{w~|~w$ je přijat RZA A$\}\subseteq\Sigma^\ast$ nazýváme \textnormal{jazyk přijímaný RZA} A.
\end{definice1}

Nyní si vyzkoušíme sazbu vět a důkazů opět s použitím
balíku \texttt{amsthm}.

\begin{veta}
Třída jazyků, které jsou přijímány ZA, odpovídá
\textnormal{bezkontextovým jazykům}.
\end{veta}

\begin{proof}
V důkaze vyjdeme z Definice \ref{def1} a \ref{def2}.
\end{proof}

\section{Rovnice a odkazy}
Složitější matematické formulace sázíme mimo plynulý
text. Lze umístit několik výrazů na jeden řádek, ale pak je
třeba tyto vhodně oddělit, například příkazem \verb!\quad!.

$$\sqrt[i]{x_i^3} \quad \textnormal{kde}~x_i~
\textnormal{je}~i\textnormal{-té sudé číslo splňující}\quad x_i^{x_i^{i^2}+2}\leq y_i^{x_i^4}
$$

V rovnici (\ref{rovnice1}) jsou využity tři typy závorek s různou
explicitně definovanou velikostí.

\begin{eqnarray}
\label{rovnice1}
x & = & \bigg[\Big\lbrace\big[a + b\big]* c\Big\rbrace^d\oplus 2\bigg]^{3/2}
\\
y & = & \lim_{x\rightarrow\infty} \frac{\frac{1}{\log_{10}x}}{\sin^2x + \cos^2x}
\nonumber
\end{eqnarray}
V této větě vidíme, jak vypadá implicitní vysázení limity lim$_{n\rightarrow\infty}~f(n)$ v normálním odstavci textu. Podobně
je to i s dalšími symboly jako $\prod_{i=1}^n 2^i$ či $\bigcap_{A\in B}A.$ V případě vzorců $\lim\limits_{ n\rightarrow\infty}~f(n)$ a $\prod\limits_{i=1}^n~2^i$
jsme si vynutili méně úspornou sazbu příkazem \verb!\limits!.

\begin{eqnarray}
\label{rovnice2}
\int_b^a g(x)~\textnormal{d}x & = & - \int\limits_a^b f(x)~\textnormal{d}x
\end{eqnarray}

\section{Matice}
Pro sázení matic se velmi často používá prostředí \texttt{array}
a závorky (\verb!\left, \right!).
$$
\left(\begin{array}{ccc}
    a - b & \widehat{\xi + \omega} & \pi \\
    \Vec{\textbf{a}} & \overleftrightarrow{AC} & \hat\beta
    \end{array}\right) = 1 \Longleftrightarrow \mathcal{Q} = \mathbb{R}$$

$$
\textbf{A} = \left|\left|\begin{array}{cccc}
a_{11} & a_{12} & \dots & a_{1n}\\
a_{21} & a_{22} & \dots & a_{2n}\\
\vdots & \vdots &\ddots & \vdots\\
a_{m1} & a_{m2} & \dots & a_{mn}
\end{array}\right|\right| = \left|
\begin{array}{cc} t & u \\ v & w\end{array}\right|= tw - uv
$$

Prostředí \texttt{array} lze úspěšně využít i jinde.

$$
\binom{n}{k} = \left\{ \begin{array}{c l}
    0 & \textnormal{pro}~k < 0~\textnormal{nebo}~k > n\\
    \frac{n!}{k!(n-k)!} & \textnormal{pro}\ 0 \leq k \leq n.
\end{array}\right.
$$
\end{document}