\documentclass[a4paper,12pt]{article}
\usepackage[utf8]{inputenc}
\usepackage[left=2cm,text={17cm,24cm},top=3cm]{geometry}
\usepackage[czech]{babel}
\usepackage[unicode, hidelinks]{hyperref}
\usepackage{times}


\begin{document}
\begin{titlepage}
\begin{center}
\huge
\textsc{VYSOKÉ UČENÍ TECHNICKÉ V BRNĚ\\
FAKULTA INFORMAČNÍCH TECHNOLOGIÍ
}
\vspace{\stretch{0,381966}}
\\
\large
Typografie a publikovaní – 4. projekt\\
\LARGE
Bibliografické citace 
\vspace{\stretch{0,618034}}
\end{center}
{\large \today \hfill
Maxim Plička(xplick04)}
\end{titlepage}

\section{\LaTeX}
Je maker balík programu \TeX, se kterým můžeme vytvářet různé texty. Tyto díla můžeme upravovat pomocí různých předdefinovaných balíčků. \LaTeX~je spíše určen pro rozsáhlejší a komplexnější dokumenty viz \cite{pavelka}. Podbrobněji si můžeme o \LaTeX u přečíst \cite{Simecek}, nebo také \cite{Davidek}.

\subsection{Hlavní výhody a nevýhody}
\textbf{Výhody:}
\begin{itemize}
    \item Výsledná kvalita dokumentu.
    \item Nezávislost na OS.
    \item Lze bezplatně využívat.
\end{itemize}
\textbf{Nevýhody:}
\begin{itemize}
    \item Čas, který musíte vynaložit pro naučení se \LaTeX u.\footnote{Pokud máte zájem naučit se sázet v \LaTeX u, můžete tak učinit zde\cite{Rybicka}}
    \item Latex není \uv{WYSIWYG}, tudíž neuvidíte změny v textu dynamicky.
\end{itemize}


\subsection{Typy editorů, které můžeme pro \LaTeX~využívat}
\begin{itemize}
    \item \textbf{Offilne editory} -- musíme je nainstalovat na počítač a následně s nimi můžeme pracovat. Představu o těchto editorech by vám mohl vytvořit \cite{Stastny}.
    \item \textbf{Online editory} -- stačí se zaregistrovat a následně se vám na účet ukládají veškeré úpravy, většinou v sobě mají už zabudovaný make. O tomto téma se více zabývá \cite{Sokol}.
    \item \textbf{Pomocné editory pro \LaTeX} -- tyto editory lze využít pro ulehčení práce a vygenerování kódu do \LaTeX u, aniž byste museli napsat samotný kód viz \cite{oksuzozcan}.
\end{itemize}
\subsection{Sazba matematických výrazů}
Sazba matematických výrazů v latexu je velmi propracovaná, proto toto prostředí začal nedávno používat i známější Word viz \cite{Matthews}. Pro sazbu matematických výrazů se nejčastěji využívá \LaTeX ový mód \texttt{math} a \texttt{displaymath}, o hlavních rozdílech se můžete dočíst v \cite{Olsak}.
Dále můžeme využít také například matematické prostředí \texttt{eqnarray}, které automaticky vytvoří číslované a zarovnané výrazy. Při sázení těchto výrazů by Vám mohlo pomoci \cite{Gratzer}.

Ukázka \texttt{eqnarray}:

\begin{eqnarray}
    \int_b^a g(x)~\textnormal{d}x & = & - \int\limits_a^b f(x)~\textnormal{d}x
\end{eqnarray}

\newpage
\renewcommand{\refname}{Použité zdroje}
\bibliographystyle{czechiso}
\bibliography{proj4}


\end{document}